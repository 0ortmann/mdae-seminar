\section{Sequence Freeze}

\subsection{Zielsetzung}
\begin{frame}
    \frametitle{\insertsection}
    \underline{Ziel: Online Algorithmus für MST}
    \vspace{1em}
    \begin{itemize}
        \itemsep\setlength{.8em}
        \item $(1+\epsilon)$-competitive
        \item Armortisiertes Budget $O(\frac{1}{\epsilon}\log\frac{1}{\epsilon})$
    \end{itemize}
    \vspace{1em}

    \underline{Abschluss:}\\
    \vspace{1em}
    Präsentierter Algorithmus liefert best mögliches Ergebnis (logarithmisch gemessen)
\end{frame}

\begin{frame}
    \frametitle{\insertsection}
    \begin{theorem}
        \vspace{1em}
        Jeder $(1+\epsilon)$-competitive Algorithmus für das online MST Problem benötigt ein armortisiertes recourse Budget von $\Omega(\frac{1}{\epsilon})$
        \vspace{1em}
    \end{theorem}
    \vspace{1em}
    \begin{theorem}
        \vspace{1em}
        Es existiert ein $(1+\epsilon)$-competitiver Algorithmus für das online MST Problem mit einem armortisierten recourse Budget von $O(\frac{1}{\epsilon}\log\frac{1}{\epsilon})$
        \vspace{1em}
    \end{theorem}
    Beweise siehe \cite{recourse2016}
\end{frame}

\subsection{Freezing Rules}
\begin{frame}
    \frametitle{\insertsubsection}
    \underline{Finde Balance zwischen Anzahl Swaps $(\leq k)$ und Nutzen}\\
    \vspace{1em}
    \begin{block}{Rule 1}
        Verhindert das Swappen von Kanten, die zu einem Subgrafen gehören, der an den Gesamtkosten des MST keinen großen Einfluss hat. -- „Kein Swap wenn indirekter Nutzen zu gering. Andere Teilgrafen sind viel einflussreicher”.
    \end{block}
    \vspace{1em}
    \begin{block}{Rule 2}
        Verhindert das Entfernen von Kanten mit geringen Kosten. -- „Kein Swap, wenn direkter Nutzen zu gering”.
    \end{block}
\end{frame}

\subsection{Algorithm Sequence Freeze}
\begin{frame}
    \frametitle{\insertsection}
    \underline{Wie simple Greedy Algorithmus mit Einschränkungen}\\
    \vspace{1em}
    Iteration $t$: $T_t = T_{t-1} \cup \{$„günstigste neue Kante”$\}$\\
    \vspace{1em}
    Betrachte Kanten $(f, h) \in (E_t \setminus T_t) \times T_t$\\
    Swap wenn nur, wenn..:
    \vspace{1em}
    \begin{itemize}
        \itemsep\setlength{1em}
        \item $(T_t \cup \{f\}) \setminus \{h\}$ bleibt ein Tree
        \item $c(h) > (1+\epsilon) \cdot c(f)$
        \item \emph{Rule 1} hält
        \item \emph{Rule 2} hält
    \end{itemize}
\end{frame}

% TODO: backup slide bauen mit den exakten formeln pro Rule.

%\subsection{Ein Lower Bound}
%\subsection{Freezes vor Swaps}