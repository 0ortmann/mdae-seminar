\documentclass{beamer}

\usepackage[ngerman]{babel}
\usepackage[utf8]{inputenc}
\usepackage[T1]{fontenc}
\usepackage{lmodern}
\usepackage{listings}

\usepackage{etoolbox}


\usetheme{CambridgeUS}
\usecolortheme{crane}

% Vor jedem neuen Abschnitt (\section) eine Folie mit Gliederung für diesen Abschnitt zeigen
\AtBeginSection[]{
    \ifnumcomp{\value{section}}{=}{1}{}{
        \frame{\frametitle{\insertsection}\tableofcontents[currentsection,hideothersubsections]}
    }
}

\title[MDAE: Recourse -- Online MST and TSP]{Methoden des Algorithmen Designs}
\subtitle{The Power of Recourse for Online MST and TSP}
\author{Felix~Ortmann}
\institute{Universität Hamburg}
\beamertemplatenavigationsymbolsempty
\setbeamertemplate{theorems}[numbered]

\begin{document}

\maketitle

\begin{frame}{Gliederung}
  \tableofcontents[hideallsubsections]
\end{frame}

\section{Einleitung}
\section{Definitionen}
\subsection{MST und TSP}

\begin{frame}
    \frametitle{\insertsubsection}
    \underline{Graf $G = (V, E)$; $G$ ist vollständig}
    \vspace{.7em}
    \begin{block}{Minimal Spanning Tree (MST)}
        \vspace{.7em}
        \begin{itemize}
            \itemsep\setlength{.8em}
            \item Verbinde alle Knoten
            \item Azyklisch
        \end{itemize}
        Gesucht: kleinste (metrische) Verbindung aller Knoten
    \end{block}
    \vspace{.7em}
    \begin{block}{Traveling Salesman Problem (TSP)}
        \vspace{.7em}
        \begin{itemize}
            \itemsep\setlength{.8em}
            \item Besuche jeden Knoten exakt einmal
            \item Zyklisch
        \end{itemize}
        Gesucht: kürzeste (metrische) Route
    \end{block}
\end{frame}

\begin{frame}
    \frametitle{\insertsubsection}
    Minimal Spanning Tree (MST)
    \begin{figure}
        \centering
        \includegraphics[width=.52\linewidth]{./img/mst.pdf}
        \caption{Minimal Spanning Tree\footnote{\url{https://commons.wikimedia.org/wiki/File:Minimum_spanning_tree.svg}}}
    \end{figure}
\end{frame}



\begin{frame}
    \frametitle{\insertsubsection}
    Traveling Salesman Problem (TSP)
    \begin{figure}
        \centering
        \includegraphics[width=.42\linewidth]{./img/tsp.png}
        \caption{TSP Tour\footnote{\url{https://commons.wikimedia.org/wiki/File:Tsp_opt.png}}}
    \end{figure}
\end{frame}
\subsection{Online Problemstellung}

\begin{frame}
    \frametitle{\insertsubsection}
    „Nach und nach Knoten und zugehörige Kanten aufdecken. Problemstellung (MST bzw. TSP) in jedem Zug auf Teilgraf möglichst gut bearbeiten.“
    \vspace{1em}
    \begin{itemize}
        \itemsep\setlength{.8em}
        \item $\sigma = v_0, v_1, ...$ wobei $v_t$: \emph{Vertex}
        \item $\forall s \leq t-1: v_tv_s$ Kanten werden zusammen mit Knoten $t$ bekannt
        \item Kantengewichte erfüllen Dreiecksungleichung ($\rightarrow$ metrisch)
        \item Iteration $t$:
        \vspace{.4em}
        \begin{itemize}
            \itemsep\setlength{.4em}
            \item Teilgraf $G_t = (V_t, E_t)$ 
            \item $V_t = \{v_0,...,v_t\}$
            \item $E_t = V_t \times V_t \rightarrow$ Graf ist „complete“
        \end{itemize}
    \end{itemize}
\end{frame}

\begin{frame}
    \frametitle{Online Ziele}
    \begin{block}{Online MST}
        \vspace{1em}
        \begin{itemize}
            \itemsep\setlength{.8em}
            \item Kanten Teilmenge $T_t$
            \item $T_t$ ist MST zu $G_t$
        \end{itemize}
        \vspace{.5em}
    \end{block}
    \vspace{1em}
    \begin{block}{Online TSP}
        \vspace{1em}
        \begin{itemize}
            \itemsep\setlength{.8em}
            \item Tour durch $G_t$: $Q_t$
            \item $Q_t$ ist TSP Tour auf $G_t$ ($\approx$ minimiert traveling Kosten)
        \end{itemize}
        \vspace{.5em}
    \end{block}
\end{frame}

\subsection{Einfacher Greedy Algorithmus -- online MST}

\begin{frame}
    \frametitle{\insertsubsection}
    \underline{Vorherige Iteration: \qquad $T_{t-1}$}\\
    \vspace{1em}
    Iteration $t$:
    \begin{itemize}
        \itemsep\setlength{.8em}
        \item Decke neuen Knoten auf: $v_t$
        \item Kürzeste Verbindungskante $g_t$ zwischen $v_t$ und $T_{t-1}$
        \item $T_t = T_{t-1} \cup g_t$
        \vspace{1em}
        \item Kostenreduktion von $T_t$ -- \textit{„Swaps”}:
        \vspace{.8em}
        \begin{itemize}
        \itemsep\setlength{.8em}
            \item Füge günstige Kante hinzu ($\rightarrow$ Kreis entsteht in Teilbaum)
            \item Entferne teuerste Kante aus Kreis
            \item Wiederhole ...
        \end{itemize}
    \end{itemize}
\end{frame}

\subsection{Budget}

\begin{frame}
    \frametitle{Swaps und Recourse}
    \begin{itemize}
        \itemsep\setlength{1em}
        \item Swaps $\rightsquigarrow$ optimale Lösung
        \item Wie viele Swaps...?
    \end{itemize}
    \vspace{1em}
    \begin{block}{Recourse}
        \vspace{1em}
        \emph{Recourse} bzw. \emph{recourse Budget} -- Menge von Kanten, die pro Iteration maximal zur Lösung hinzugefügt werden.
        \vspace{1em}
    \end{block}
\end{frame}

\begin{frame}
    \frametitle{\insertsubsection}
    \underline{Iteration $t$, Budget $k$}\\
    \vspace{1em}
    \begin{block}{Striktes Budget}
        \vspace{1em}
        $\forall t \geq 1:\hspace{0.5em}|T_t \setminus T_{t-1}| \leq k$
        \vspace{1em}
    \end{block}
    \vspace{1em}
    \begin{block}{Amortisiertes Budget}
        \vspace{1em}
        $ \sum_{s=1}^{t}\hspace{0.5em}|T_s \setminus T_{s-1}| \quad \leq k \cdot t$
        \vspace{1em}
    \end{block}
\end{frame}

\section{Online MST mit armortisiertem Budget}

\subsection{Zielsetzung}
\begin{frame}
    \frametitle{\insertsubsection}
    \underline{Ziel}\\
    \vspace{1em}
    Online Algorithmus für MST mit folgenden Eigenschaften: (Theorem~\ref{thm2})\\
    \vspace{1em}
    \begin{itemize}
        \itemsep\setlength{.8em}
        \item $(1+\epsilon)$-competitive
        \item Armortisiertes Budget $O(\frac{1}{\epsilon}\log\frac{1}{\epsilon})$
    \end{itemize}
    \vspace{1em}

    \underline{Abschluss}\\
    \vspace{1em}
    Präsentierter Algorithmus liefert best mögliches Ergebnis, logarithmisch gemessen (Theorem~\ref{thm1})
\end{frame}

\begin{frame}
    \frametitle{\insertsection}
    \begin{theorem}[Untere Schranke]
        \label{thm1}
        \vspace{.5em}
        Jeder $(1+\epsilon)$-competitive Algorithmus für das online MST Problem benötigt ein armortisiertes recourse Budget von $\Omega(\frac{1}{\epsilon})$
        \vspace{1em}
    \end{theorem}
    \vspace{1em}
    Beweis siehe \cite{recourse2016}
    \vspace{1em}

    \begin{theorem}[Obere Schranke]
        \label{thm2}
        \vspace{.5em}
        Es existiert ein $(1+\epsilon)$-competitiver Algorithmus für das online MST Problem mit einem armortisierten recourse Budget von $O(\frac{1}{\epsilon}\log\frac{1}{\epsilon})$
        \vspace{1em}
    \end{theorem}
\end{frame}

\subsection{Freezing Rules}
\begin{frame}
    \frametitle{\insertsubsection}
    \underline{Finde Balance zwischen Anzahl Swaps $(\leq k)$ und Nutzen}\\
    \vspace{1em}
    \begin{block}{Rule 1}
        Verhindert das Swappen von Kanten, die zu einem Subgrafen gehören, der an den Gesamtkosten des MST keinen großen Einfluss hat. -- „Kein Swap wenn indirekter Nutzen zu gering. Andere Teilgrafen sind viel einflussreicher”.
    \end{block}
    \vspace{1em}
    \begin{block}{Rule 2}
        Verhindert das Entfernen von Kanten mit geringen Kosten. -- „Kein Swap, wenn direkter Nutzen zu gering”.
    \end{block}
\end{frame}

\begin{frame}
    \frametitle{Rule 1 und 2 Formal}
    FIXME: hier mit allen nötigen (so schmal wie möglich die beiden Rules formal definieren)
\end{frame}

\subsection{Algorithm Sequence Freeze}
\begin{frame}
    \frametitle{\insertsubsection}
    \underline{Wie simple Greedy Algorithmus mit Einschränkungen}\\
    \vspace{1em}
    Iteration $t$: $T_t = T_{t-1} \cup \{$„günstigste neue Kante”$\}$\\
    \vspace{1em}
    Betrachte Kanten $(f, h) \in (E_t \setminus T_t) \times T_t$\\
    Swap wenn nur, wenn..:
    \vspace{1em}
    \begin{itemize}
        \itemsep\setlength{1em}
        \item $(T_t \cup \{f\}) \setminus \{h\}$ bleibt ein Tree
    \end{itemize}
    \begin{enumerate}
        \itemsep\setlength{1em}
        \item $c(h) > (1+\epsilon) \cdot c(f)$
        \item FIXME: hier formal (\emph{Rule 1} hält)
        \item FIXME: hier formal (\emph{Rule 2} hält)
    \end{enumerate}
\end{frame}

\subsection{Beweisskizze zu Theorem (2)}
\begin{frame}
    \frametitle{\insertsubsection}
    Competitive Analysis des Algoithmus\\
    \vspace{1em}
    \begin{block}{$(1+\epsilon)$-competitive}
    \vspace{1em}
        \begin{itemize}
            \itemsep\setlength{1em}
            \item \emph{Cond. 1} \& \emph{Cond. 3} -- Kosten steigen maximal um $(1+3\epsilon)$ pro Iteration $\rightarrow (1+O(\epsilon))$-competitive
            \item \emph{Cond. 2} -- Durch Weglassen „kostengünstiger” Swaps wird die Gesamtlösung im Vergleich zu $OPT_t$ maximal $O(\epsilon OPT_t)$ schlechter
        \end{itemize}
    \vspace{1em}
    \end{block}
\end{frame}

\subsection{Beweisskizze zu Theorem (2)}
\begin{frame}
    \frametitle{\insertsubsection}
    Armotisierter Upper Bound für Budget
    \vspace{1em}
    \begin{block}{Upper Bound auf Swap Sequenz Länge}
        \vspace{1em}
        \emph{Cond. 1} -- Nur Swap wenn Kostengewinn $> (1+\epsilon)$:\\Maximale Swapanzahl $\implies \log_{1+\epsilon}c(g_s^0) - \log_{1+\epsilon}c(g_s^{i(s)-1}) + 1$
        \vspace{1em}
    \end{block}
    \vspace{1em}
    \begin{block}{Bound auf Swap Kosten}
        \vspace{1em}
        \emph{Cond. 3} -- Nur Swap wenn entfernte Kante Threshold übersteigt \\
        \emph{FIXME FORMAL}
        \vspace{1em}
    \end{block}
\end{frame}

\subsection{Recap}
\begin{frame}
    \frametitle{\insertsection}
    \underline{Ergebnis}\\
    \vspace{1em}
    Online Algorithmus für MST mit folgenden bewiesenen Eigenschaften:\\
    \vspace{1em}
    \begin{itemize}
        \itemsep\setlength{.8em}
        \item $(1+\epsilon)$-competitive
        \item \underline{Armortisiertes} Budget $O(\frac{1}{\epsilon}\log\frac{1}{\epsilon})$
    \end{itemize}
    \vspace{1em}
    Da jeder online Algorithmus für MST mit obigen Eigenschaften mindestens ein Budget von $\Omega(\frac{1}{\epsilon})$ benötigt (vgl. Theorem~\ref{thm1}), liefert \emph{Sequence Freeze} das (logarithmisch) bestmögliche online Ergebnis.
\end{frame}
\section{Zusammenfassung}

\begin{frame}
	\frametitle{\insertsection}
	\begin{itemize}
		\itemsep\setlength{1em}
		\item Standard online MST: Beste bekannte competitive Ratio $\Theta{\log n}$
		\item Neu Erkenntnis mit \cite{recourse2016}:\\Recourse verbessert die competitive Ratio zu \underline{$(1+\epsilon)$} (amortisiert)
		\item Verändertes Shortcutting Verfahren
		\item Online MST $\xrightarrow{\text{det. Verfahren}}$ online TSP\\
		Bewiesene Upper Bounds:
		\vspace{1em}
		\begin{itemize}
			\itemsep\setlength{.7em}
			\item Erhöht competitive Ratio um 2
			\item Erhöht Budget um 4
		\end{itemize}
		\item Online TSP $(2+\epsilon)$-competitive mit armortisiertem Budget $O(\frac{1}{\epsilon}\log\frac{1}{\epsilon})$
	\end{itemize}
\end{frame}

\section{Literatur}

\nocite{*}

\begin{frame}[allowframebreaks]{\insertsubsection}
    \begingroup
    \small
    \beamertemplatebookbibitems
    \bibliographystyle{apalike}
    \bibliography{bib}
    \endgroup
\end{frame}

\end{document}
