\section{Online MST mit amortisiertem Budget}

\subsection{Zielsetzung}
\begin{frame}
    \frametitle{\insertsubsection}
    \underline{Ziel}\\
    \vspace{1em}
    Online Algorithmus für MST mit folgenden Eigenschaften $(\epsilon > 0)$: (Theorem~\ref{thm2})\\
    \vspace{1em}
    \begin{itemize}
        \itemsep\setlength{.8em}
        \item $(1+\epsilon)$-competitive
        \item Amortisiertes Budget $O(\frac{1}{\epsilon}\log\frac{1}{\epsilon})$
    \end{itemize}
    \vspace{1em}

    \underline{Abschluss}\\
    \vspace{1em}
    Präsentierter Algorithmus liefert best mögliches Ergebnis, logarithmisch gemessen (Theorem~\ref{thm1})
\end{frame}

\begin{frame}
    \frametitle{\insertsection}
    \begin{theorem}[Lower Bound]
        \label{thm1}
        \vspace{.5em}
        Jeder $(1+\epsilon)$-competitive Algorithmus für das online MST Problem benötigt ein amortisiertes recourse Budget von $\Omega(\frac{1}{\epsilon})$
        \vspace{1em}
    \end{theorem}
    Beweis siehe \cite{recourse2016}
    \vspace{1em}

    \begin{theorem}[Upper Bound]
        \label{thm2}
        \vspace{.5em}
        Es existiert ein $(1+\epsilon)$-competitiver Algorithmus für das online MST Problem mit einem amortisierten recourse Budget von $O(\frac{1}{\epsilon}\log\frac{1}{\epsilon})$
        \vspace{1em}
    \end{theorem}
\end{frame}

\subsection{Freezing Rules}


\begin{frame}
    \frametitle{\insertsubsection}
    \underline{Iteration $t$:}
    \vspace{.8em}
    \begin{itemize}
        \itemsep\setlength{1em}
        \item $OPT_t^{max}$: bestes (von unserem Algorithmus) erreichbares Ergebnis
        \item $\ell(t)$: Iteration mit den meisten Swaps (vor dieser Iteration)
        \item $\ell(t) \leq (t-1)$. $OPT_{\ell(t)}^{max} \leq \epsilon \cdot OPT_t^{max}$
        \item Greedy Kante $g_s^0$: In Iteration $s$ greedy hinzugefügt
        \item Swap: $g_s^1$ sei Kante, die $g_s^0$ ersetzt
        \item Wurde in später ggf. geswapped, momentane Ersetzung: $g_s^{i(s)}$
    \end{itemize}
\end{frame}

\begin{frame}
    \frametitle{\insertsubsection}
    \begin{block}{Rule 1}
        \vspace{.6em}
        Freeze Sequenz $(g_s^0, ..., g_s^{i(s)})$ gwd. $s \leq \ell(t)$
        \vspace{.6em}
    \end{block}
    \vspace{1em}
    \begin{block}{Rule 2}
        \vspace{.6em}
        Freeze Kante wenn der Kostengewinn kleiner als $\epsilon OPT_t^{max}/ (t-\ell(t))$ ist.
        \vspace{.6em}
    \end{block}
\end{frame}

\begin{frame}
    \frametitle{\insertsubsection~In Worten}
    \underline{Finde Balance zwischen Anzahl Swaps $(\leq k)$ und Nutzen}\\
    \vspace{1em}
    \begin{block}{Rule 1}
        Verhindert das Swappen von Kanten, die zu einem Subgrafen gehören, der an den Gesamtkosten des MST keinen großen Einfluss hat.\\
        $\implies$Wurde vor / während der größten Iteration $\ell(t)$ geswapped. „Den besten bisherigen Subgrafen (und alles was schon davor gefroren war) fassen wir nicht mehr an.“
    \end{block}
    \vspace{1em}
    \begin{block}{Rule 2}
        Verhindert das Entfernen von Kanten deren Kosten kleiner als das Mittel seit der größten Iteration sind.\\
        $\implies$„Kein Swap, wenn direkter Nutzen zu gering”.
    \end{block}
\end{frame}


\subsection{Algorithmus Sequence Freeze}
\begin{frame}
    \frametitle{\insertsubsection}
    \underline{Wie simple Greedy Algorithmus mit Einschränkungen}\\
    \vspace{1em}
    $T_0 = \emptyset$\\
    \vspace{1em}
    Iteration $t$: $T_t = T_{t-1} \cup \{$„günstigste neue Kante”$\}$\\
    \vspace{1em}
    Betrachte Kanten $(f, h) \in (E_t \setminus T_t) \times T_t$\\
    Swap wenn nur, wenn..:
    \vspace{1em}
    \begin{itemize}
        \itemsep\setlength{1em}
        \item $(T_t \cup \{f\}) \setminus \{h\}$ bleibt ein Tree
    \end{itemize}
    \begin{enumerate}
        \itemsep\setlength{1em}
        \item $c(h) > (1+\epsilon) \cdot c(f)$
        \item $h = g_s^{i(s)}$ für ein $s \geq \ell(t) + 1$ (nicht gefroren durch \emph{Rule 1})
        \item $c(h) > \epsilon\frac{OPT_t^{max}}{t-\ell(t)}$ (nicht gefroren durch \emph{Rule 2})
    \end{enumerate}
\end{frame}

\subsection{Competitive Analysis}
\begin{frame}
    \frametitle{\insertsubsection~(Skizze)}
    Competitiveness des Algoithmus im Vergleich zu OPT\\
    \vspace{.5em}
    \begin{block}{$(1+\epsilon)$-competitive}
        \vspace{.5em}
        \begin{itemize}
            \itemsep\setlength{.5em}
            \item \emph{Cond. 1} \& \emph{Cond. 3} -- Kosten steigen maximal um $(1+3\epsilon)$ pro Iteration $\rightarrow (1+O(\epsilon))$-competitive
            \item \emph{Cond. 2} -- Durch Weglassen „kostengünstiger” Swaps wird die Gesamtlösung im Vergleich zu $OPT_t$ maximal $O(\epsilon OPT_t)$ schlechter
        \end{itemize}
        \vspace{.1em}
    \end{block}
    \begin{itemize}
        \itemsep\setlength{.5em}
        \item Partitioniere Graf $T = T_{new} \cup T_{old}$ -- vor / nach längster Iteration
        \item $T_{old} := \{g_1^{i(1)}, ..., g_{\ell(t)}^{i(\ell(t))}\}$
        \item $T_{new} := \{g_{\ell(t)+1}^{i(\ell(t)+1)}, ..., g_t^{i(t)}\}$
    \end{itemize}
    Beweise Bounds für Partitionen
\end{frame}

\begin{frame}
    \frametitle{\insertsubsection~(Skizze)}
    Amortisierter Upper Bound für Budget
    \vspace{.5em}
    \begin{itemize}
        \itemsep\setlength{.5em}
        \item $k_q := |T_q \setminus T_{q-1}|$: \underline{benutztes} Budget in Iteration $q$
        \item $D_\epsilon \in O(\frac{1}{\epsilon}\log\frac{1}{\epsilon})$
    \end{itemize}
    \vspace{.5em}
    \begin{block}{Zeige $\sum_{q=1}^t k_q \leq D_\epsilon \cdot t$}
        \vspace{.5em}
        \begin{enumerate}
            \itemsep\setlength{.8em}
            \item Upper Bound auf Swap Sequenz Länge\\
            \emph{Cond. 1} -- Nur Swap wenn Kostengewinn $> (1+\epsilon)$:\\Maximale Swapanzahl $\implies \log_{1+\epsilon}c(g_s^0) - \log_{1+\epsilon}c(g_s^{i(s)-1}) + 1$
            \item Lower Bound auf Swap Kosten\\
            \emph{Cond. 3} -- Nur Swap wenn Kosten d. zu entfernenden Kante Threshold übersteigt
        \end{enumerate}
        \vspace{.1em}
    \end{block}
\end{frame}

\subsection{Rückblick}
\begin{frame}
    \frametitle{\insertsection}
    \underline{Ergebnis}\\
    \vspace{1em}
    Online Algorithmus für MST mit folgenden bewiesenen Eigenschaften:\\
    \vspace{1em}
    \begin{itemize}
        \itemsep\setlength{.8em}
        \item $(1+\epsilon)$-competitive
        \item \underline{Amortisiertes} Budget $O(\frac{1}{\epsilon}\log\frac{1}{\epsilon})$
    \end{itemize}
    \vspace{1em}
    Da jeder online Algorithmus für MST mit obigen Eigenschaften mindestens ein Budget von $\Omega(\frac{1}{\epsilon})$ benötigt (vgl. Theorem~\ref{thm1}), liefert \emph{Sequence Freeze} das (logarithmisch) bestmögliche online Ergebnis.
\end{frame}