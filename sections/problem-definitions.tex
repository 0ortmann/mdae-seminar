\section{Definitionen}
\subsection{MST und TSP}

\begin{frame}
    \frametitle{\insertsubsection}
    \underline{Graf $G = (V, E)$; $G$ ist vollständig}
    \vspace{.7em}
    \begin{block}{Minimal Spanning Tree (MST)}
        \vspace{.7em}
        \begin{itemize}
            \itemsep\setlength{.8em}
            \item Verbinde alle Knoten
            \item Azyklisch
        \end{itemize}
        Gesucht: kleinste (metrische) Verbindung aller Knoten
    \end{block}
    \vspace{.7em}
    \begin{block}{Traveling Salesman Problem (TSP)}
        \vspace{.7em}
        \begin{itemize}
            \itemsep\setlength{.8em}
            \item Besuche jeden Knoten exakt einmal
            \item Zyklisch
        \end{itemize}
        Gesucht: kürzeste (metrische) Route
    \end{block}
\end{frame}

\begin{frame}
    \frametitle{\insertsubsection}
    Minimal Spanning Tree (MST)
    \begin{figure}
        \centering
        \includegraphics[width=.52\linewidth]{./img/mst.pdf}
        \caption{Minimal Spanning Tree\footnote{\url{https://commons.wikimedia.org/wiki/File:Minimum_spanning_tree.svg}}}
    \end{figure}
\end{frame}



\begin{frame}
    \frametitle{\insertsubsection}
    Traveling Salesman Problem (TSP)
    \begin{figure}
        \centering
        \includegraphics[width=.42\linewidth]{./img/tsp.png}
        \caption{TSP Tour\footnote{\url{https://commons.wikimedia.org/wiki/File:Tsp_opt.png}}}
    \end{figure}
\end{frame}
\subsection{Online Problemstellung}

\begin{frame}
    \frametitle{\insertsubsection}
    „Nach und nach Knoten und zugehörige Kanten aufdecken. Problemstellung (MST bzw. TSP) in jedem Zug auf Teilgraf möglichst gut bearbeiten.“
    \vspace{1em}
    \begin{itemize}
        \itemsep\setlength{.8em}
        \item $\sigma = v_0, v_1, ...$ wobei $v_t$: \emph{Vertex}
        \item $\forall s \leq t-1: v_tv_s$ Kanten werden zusammen mit Knoten $t$ bekannt
        \item Kantengewichte erfüllen Dreiecksungleichung ($\rightarrow$ metrisch)
        \item Iteration $t$:
        \vspace{.4em}
        \begin{itemize}
            \itemsep\setlength{.4em}
            \item Teilgraf $G_t = (V_t, E_t)$ 
            \item $V_t = \{v_0,...,v_t\}$
            \item $E_t = V_t \times V_t \rightarrow$ Graf ist „complete“
        \end{itemize}
    \end{itemize}
\end{frame}

\begin{frame}
    \frametitle{Online Ziele}
    \begin{block}{Online MST}
        \vspace{1em}
        \begin{itemize}
            \itemsep\setlength{.8em}
            \item Kanten Teilmenge $T_t$
            \item $T_t$ ist MST zu $G_t$
        \end{itemize}
        \vspace{.5em}
    \end{block}
    \vspace{1em}
    \begin{block}{Online TSP}
        \vspace{1em}
        \begin{itemize}
            \itemsep\setlength{.8em}
            \item Tour durch $G_t$: $Q_t$
            \item $Q_t$ ist TSP Tour auf $G_t$ ($\approx$ minimiert traveling Kosten)
        \end{itemize}
        \vspace{.5em}
    \end{block}
\end{frame}

\subsection{Einfacher Greedy Algorithmus -- online MST}

\begin{frame}
    \frametitle{\insertsubsection}
    \underline{Vorherige Iteration: \qquad $T_{t-1}$}\\
    \vspace{1em}
    Iteration $t$:
    \begin{itemize}
        \itemsep\setlength{.8em}
        \item Decke neuen Knoten auf: $v_t$
        \item Kürzeste Verbindungskante $g_t$ zwischen $v_t$ und $T_{t-1}$
        \item $T_t = T_{t-1} \cup g_t$
        \vspace{1em}
        \item Kostenreduktion von $T_t$ -- \textit{„Swaps”}:
        \vspace{.8em}
        \begin{itemize}
        \itemsep\setlength{.8em}
            \item Füge günstige Kante hinzu ($\rightarrow$ Kreis entsteht in Teilbaum)
            \item Entferne teuerste Kante aus Kreis
            \item Wiederhole ...
        \end{itemize}
    \end{itemize}
\end{frame}

\subsection{Budget}

\begin{frame}
    \frametitle{Swaps und Recourse}
    \begin{itemize}
        \itemsep\setlength{1em}
        \item Swaps $\rightsquigarrow$ optimale Lösung
        \item Wie viele Swaps...?
    \end{itemize}
    \vspace{1em}
    \begin{block}{Recourse}
        \vspace{1em}
        \emph{Recourse} bzw. \emph{recourse Budget} -- Menge von Kanten, die pro Iteration maximal zur Lösung hinzugefügt werden.
        \vspace{1em}
    \end{block}
\end{frame}

\begin{frame}
    \frametitle{\insertsubsection}
    \underline{Iteration $t$, Budget $k$}\\
    \vspace{1em}
    \begin{block}{Striktes Budget}
        \vspace{1em}
        $\forall t \geq 1:\hspace{0.5em}|T_t \setminus T_{t-1}| \leq k$
        \vspace{1em}
    \end{block}
    \vspace{1em}
    \begin{block}{Amortisiertes Budget}
        \vspace{1em}
        $ \sum_{s=1}^{t}\hspace{0.5em}|T_s \setminus T_{s-1}| \quad \leq k \cdot t$
        \vspace{1em}
    \end{block}
\end{frame}
