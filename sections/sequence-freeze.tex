\section{Online MST mit armortisiertem Budget}

\subsection{Zielsetzung}
\begin{frame}
    \frametitle{\insertsubsection}
    \underline{Ziel}\\
    \vspace{1em}
    Online Algorithmus für MST mit folgenden Eigenschaften: (Theorem~\ref{thm2})\\
    \vspace{1em}
    \begin{itemize}
        \itemsep\setlength{.8em}
        \item $(1+\epsilon)$-competitive
        \item Armortisiertes Budget $O(\frac{1}{\epsilon}\log\frac{1}{\epsilon})$
    \end{itemize}
    \vspace{1em}

    \underline{Abschluss}\\
    \vspace{1em}
    Präsentierter Algorithmus liefert best mögliches Ergebnis, logarithmisch gemessen (Theorem~\ref{thm1})
\end{frame}

\begin{frame}
    \frametitle{\insertsection}
    \begin{theorem}
        \label{thm1}
        \vspace{1em}
        Jeder $(1+\epsilon)$-competitive Algorithmus für das online MST Problem benötigt ein armortisiertes recourse Budget von $\Omega(\frac{1}{\epsilon})$
        \vspace{1em}
    \end{theorem}
    \vspace{1em}
    Beweis siehe \cite{recourse2016}
    \vspace{1em}

    \begin{theorem}
        \label{thm2}
        \vspace{1em}
        Es existiert ein $(1+\epsilon)$-competitiver Algorithmus für das online MST Problem mit einem armortisierten recourse Budget von $O(\frac{1}{\epsilon}\log\frac{1}{\epsilon})$
        \vspace{1em}
    \end{theorem}
\end{frame}

\subsection{Freezing Rules}
\begin{frame}
    \frametitle{\insertsubsection}
    \underline{Finde Balance zwischen Anzahl Swaps $(\leq k)$ und Nutzen}\\
    \vspace{1em}
    \begin{block}{Rule 1}
        Verhindert das Swappen von Kanten, die zu einem Subgrafen gehören, der an den Gesamtkosten des MST keinen großen Einfluss hat. -- „Kein Swap wenn indirekter Nutzen zu gering. Andere Teilgrafen sind viel einflussreicher”.
    \end{block}
    \vspace{1em}
    \begin{block}{Rule 2}
        Verhindert das Entfernen von Kanten mit geringen Kosten. -- „Kein Swap, wenn direkter Nutzen zu gering”.
    \end{block}
\end{frame}

\subsection{Algorithm Sequence Freeze}
\begin{frame}
    \frametitle{\insertsubsection}
    \underline{Wie simple Greedy Algorithmus mit Einschränkungen}\\
    \vspace{1em}
    Iteration $t$: $T_t = T_{t-1} \cup \{$„günstigste neue Kante”$\}$\\
    \vspace{1em}
    Betrachte Kanten $(f, h) \in (E_t \setminus T_t) \times T_t$\\
    Swap wenn nur, wenn..:
    \vspace{1em}
    \begin{itemize}
        \itemsep\setlength{1em}
        \item $(T_t \cup \{f\}) \setminus \{h\}$ bleibt ein Tree
    \end{itemize}
    \begin{enumerate}
        \itemsep\setlength{1em}
        \item $c(h) > (1+\epsilon) \cdot c(f)$
        \item \emph{Rule 1} hält
        \item \emph{Rule 2} hält
    \end{enumerate}
\end{frame}

\subsection{Beweisskizze zu Theorem (2)}
\begin{frame}
    \frametitle{\insertsubsection}
    Competitive Analysis des Algoithmus\\
    \vspace{1em}
    \begin{block}{$(1+\epsilon)$-competitive}
    \vspace{1em}
        \begin{itemize}
            \itemsep\setlength{1em}
            \item \emph{Cond. 1} \& \emph{Cond. 3} -- Kosten steigen maximal um $(1+3\epsilon)$ pro Iteration $\rightarrow (1+O(\epsilon))$-competitive
            \item \emph{Cond. 2} -- Durch Weglassen „kostengünstiger” Swaps wird die Gesamtlösung im Vergleich zu $OPT_t$ maximal $O(\epsilon OPT_t)$ schlechter
        \end{itemize}
    \vspace{1em}
    \end{block}
\end{frame}

\subsection{Beweisskizze zu Theorem (2)}
\begin{frame}
    \frametitle{\insertsubsection}
    Armotisierter Upper Bound für Budget
    \vspace{1em}
    \begin{block}{Upper Bound auf Swap Sequenz Länge}
        \vspace{1em}
        \emph{Cond. 1} -- Nur Swap wenn Kostengewinn $> (1+\epsilon)$:\\Maximale Swapanzahl $\implies \log_{1+\epsilon}c(g_s^0) - \log_{1+\epsilon}c(g_s^{i(s)-1}) + 1$
        \vspace{1em}
    \end{block}
    \vspace{1em}
    \begin{block}{Bound auf Swap Kosten}
        \vspace{1em}
        \emph{Cond. 3} -- Nur Swap wenn entfernte Kante Threshold übersteigt \\
        \emph{FIXME}
        \vspace{1em}
    \end{block}
\end{frame}

\subsection{Recap}
\begin{frame}
    \frametitle{\insertsection}
    \underline{Ergebnis}\\
    \vspace{1em}
    Online Algorithmus für MST mit folgenden bewiesenen Eigenschaften:\\
    \vspace{1em}
    \begin{itemize}
        \itemsep\setlength{.8em}
        \item $(1+\epsilon)$-competitive
        \item \underline{Armortisiertes} Budget $O(\frac{1}{\epsilon}\log\frac{1}{\epsilon})$
    \end{itemize}
    \vspace{1em}
    Da jeder online Algorithmus für MST mit obigen Eigenschaften mindestens ein Budget von $\Omega(\frac{1}{\epsilon})$ benötigt, liefert \emph{Sequence Freeze} das (logarithmisch) bestmögliche online Ergebnis.
\end{frame}