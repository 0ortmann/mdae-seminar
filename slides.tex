\documentclass{beamer}

\usepackage[ngerman]{babel}
\usepackage[utf8]{inputenc}
\usepackage[T1]{fontenc}
\usepackage{lmodern}
\usepackage{listings}

\usepackage{etoolbox}


\usetheme{CambridgeUS}
\usecolortheme{crane}

% Vor jedem neuen Abschnitt (\section) eine Folie mit Gliederung für diesen Abschnitt zeigen
\AtBeginSection[]{
    \ifnumcomp{\value{section}}{=}{1}{}{
        \frame{\frametitle{\insertsection}\tableofcontents[sectionstyle=show/hide,subsectionstyle=show/show/hide]}
    }
}

\title[MDAE: Recourse -- Online MST and TSP]{Methoden des Algorithmen Designs}
\subtitle{The Power of Recourse for Online MST and TSP}
\author{Felix~Ortmann}
\institute{Universität Hamburg}
\beamertemplatenavigationsymbolsempty
\setbeamertemplate{theorems}[numbered]

\begin{document}

\maketitle

\begin{frame}{Gliederung}
  \tableofcontents[hideallsubsections]
\end{frame}

\section{Einleitung}
\section{Definitionen}
\subsection{MST und TSP}

\begin{frame}
    \frametitle{\insertsubsection}
    \underline{Graf $G = (V, E)$; $G$ ist vollständig}
    \vspace{.7em}
    \begin{block}{Minimal Spanning Tree (MST)}
        \vspace{.7em}
        \begin{itemize}
            \itemsep\setlength{.8em}
            \item Verbinde alle Knoten
            \item Azyklisch
        \end{itemize}
        Gesucht: kleinste (metrische) Verbindung aller Knoten
    \end{block}
    \vspace{.7em}
    \begin{block}{Traveling Salesman Problem (TSP)}
        \vspace{.7em}
        \begin{itemize}
            \itemsep\setlength{.8em}
            \item Besuche jeden Knoten exakt einmal
            \item Zyklisch
        \end{itemize}
        Gesucht: kürzeste (metrische) Route
    \end{block}
\end{frame}

\begin{frame}
    \frametitle{\insertsubsection}
    Minimal Spanning Tree (MST)
    \begin{figure}
        \centering
        \includegraphics[width=.52\linewidth]{./img/mst.pdf}
        \caption{Minimal Spanning Tree\footnote{\url{https://commons.wikimedia.org/wiki/File:Minimum_spanning_tree.svg}}}
    \end{figure}
\end{frame}



\begin{frame}
    \frametitle{\insertsubsection}
    Traveling Salesman Problem (TSP)
    \begin{figure}
        \centering
        \includegraphics[width=.42\linewidth]{./img/tsp.png}
        \caption{TSP Tour\footnote{\url{https://commons.wikimedia.org/wiki/File:Tsp_opt.png}}}
    \end{figure}
\end{frame}
\subsection{Online Problemstellung}

\begin{frame}
    \frametitle{\insertsubsection}
    „Nach und nach Knoten und zugehörige Kanten aufdecken. Problemstellung (MST bzw. TSP) in jedem Zug auf Teilgraf möglichst gut bearbeiten.“
    \vspace{1em}
    \begin{itemize}
        \itemsep\setlength{.8em}
        \item $\sigma = v_0, v_1, ...$ wobei $v_t$: \emph{Vertex}
        \item $\forall s \leq t-1: v_tv_s$ Kanten werden zusammen mit Knoten $t$ bekannt
        \item Kantengewichte erfüllen Dreiecksungleichung ($\rightarrow$ metrisch)
        \item Iteration $t$:
        \vspace{.4em}
        \begin{itemize}
            \itemsep\setlength{.4em}
            \item Teilgraf $G_t = (V_t, E_t)$ 
            \item $V_t = \{v_0,...,v_t\}$
            \item $E_t = V_t \times V_t \rightarrow$ Graf ist „complete“
        \end{itemize}
    \end{itemize}
\end{frame}

\begin{frame}
    \frametitle{Online Ziele}
    \begin{block}{Online MST}
        \vspace{1em}
        \begin{itemize}
            \itemsep\setlength{.8em}
            \item Kanten Teilmenge $T_t$
            \item $T_t$ ist MST zu $G_t$
        \end{itemize}
        \vspace{.5em}
    \end{block}
    \vspace{1em}
    \begin{block}{Online TSP}
        \vspace{1em}
        \begin{itemize}
            \itemsep\setlength{.8em}
            \item Tour durch $G_t$: $Q_t$
            \item $Q_t$ ist TSP Tour auf $G_t$ ($\approx$ minimiert traveling Kosten)
        \end{itemize}
        \vspace{.5em}
    \end{block}
\end{frame}

\subsection{Einfacher Greedy Algorithmus -- online MST}

\begin{frame}
    \frametitle{\insertsubsection}
    \underline{Vorherige Iteration: \qquad $T_{t-1}$}\\
    \vspace{1em}
    Iteration $t$:
    \begin{itemize}
        \itemsep\setlength{.8em}
        \item Decke neuen Knoten auf: $v_t$
        \item Kürzeste Verbindungskante $g_t$ zwischen $v_t$ und $T_{t-1}$
        \item $T_t = T_{t-1} \cup g_t$
        \vspace{1em}
        \item Kostenreduktion von $T_t$ -- \textit{„Swaps”}:
        \vspace{.8em}
        \begin{itemize}
        \itemsep\setlength{.8em}
            \item Füge günstige Kante hinzu ($\rightarrow$ Kreis entsteht in Teilbaum)
            \item Entferne teuerste Kante aus Kreis
            \item Wiederhole ...
        \end{itemize}
    \end{itemize}
\end{frame}

\subsection{Budget}

\begin{frame}
    \frametitle{Swaps und Recourse}
    \begin{itemize}
        \itemsep\setlength{1em}
        \item Swaps $\rightsquigarrow$ optimale Lösung
        \item Wie viele Swaps...?
    \end{itemize}
    \vspace{1em}
    \begin{block}{Recourse}
        \vspace{1em}
        \emph{Recourse} bzw. \emph{recourse Budget} -- Menge von Kanten, die pro Iteration maximal zur Lösung hinzugefügt werden.
        \vspace{1em}
    \end{block}
\end{frame}

\begin{frame}
    \frametitle{\insertsubsection}
    \underline{Iteration $t$, Budget $k$}\\
    \vspace{1em}
    \begin{block}{Striktes Budget}
        \vspace{1em}
        $\forall t \geq 1:\hspace{0.5em}|T_t \setminus T_{t-1}| \leq k$
        \vspace{1em}
    \end{block}
    \vspace{1em}
    \begin{block}{Amortisiertes Budget}
        \vspace{1em}
        $ \sum_{s=1}^{t}\hspace{0.5em}|T_s \setminus T_{s-1}| \quad \leq k \cdot t$
        \vspace{1em}
    \end{block}
\end{frame}

\section{Online MST mit amortisiertem Budget}

\subsection{Zielsetzung}
\begin{frame}
    \frametitle{\insertsubsection}
    \underline{Ziel}\\
    \vspace{1em}
    Online Algorithmus für MST mit folgenden Eigenschaften: (Theorem~\ref{thm2})\\
    \vspace{1em}
    \begin{itemize}
        \itemsep\setlength{.8em}
        \item $(1+\epsilon)$-competitive
        \item Amortisiertes Budget $O(\frac{1}{\epsilon}\log\frac{1}{\epsilon})$
    \end{itemize}
    \vspace{1em}

    \underline{Abschluss}\\
    \vspace{1em}
    Präsentierter Algorithmus liefert best mögliches Ergebnis, logarithmisch gemessen (Theorem~\ref{thm1})
\end{frame}

\begin{frame}
    \frametitle{\insertsection}
    \begin{theorem}[Lower Bound]
        \label{thm1}
        \vspace{.5em}
        Jeder $(1+\epsilon)$-competitive Algorithmus für das online MST Problem benötigt ein amortisiertes recourse Budget von $\Omega(\frac{1}{\epsilon})$
        \vspace{1em}
    \end{theorem}
    Beweis siehe \cite{recourse2016}
    \vspace{1em}

    \begin{theorem}[Upper Bound]
        \label{thm2}
        \vspace{.5em}
        Es existiert ein $(1+\epsilon)$-competitiver Algorithmus für das online MST Problem mit einem amortisierten recourse Budget von $O(\frac{1}{\epsilon}\log\frac{1}{\epsilon})$
        \vspace{1em}
    \end{theorem}
\end{frame}

\subsection{Freezing Rules}


\begin{frame}
    \frametitle{\insertsubsection}
    \underline{Iteration $t$:}
    \vspace{.8em}
    \begin{itemize}
        \itemsep\setlength{1em}
        \item $OPT_t^{max}$: bestes (von unserem Algorithmus) erreichbares Ergebnis
        \item $\ell(t)$: Iteration mit den meisten Swaps (vor dieser Iteration)
        \item $\ell(t) \leq (t-1)$. $OPT_{\ell(t)}^{max} \leq \epsilon \cdot OPT_t^{max}$
        \item Greedy Kante $g_s^0$: In Iteration $s$ greedy hinzugefügt
        \item Swap: $g_s^1$ sei Kante, die $g_s^0$ ersetzt
        \item Wurde in später ggf. geswapped, momentane Ersetzung: $g_s^{i(s)}$
    \end{itemize}
\end{frame}

\begin{frame}
    \frametitle{\insertsubsection}
    \begin{block}{Rule 1}
        \vspace{.6em}
        Freeze Sequenz $(g_s^0, ..., g_s^{i(s)})$ gwd. $s \leq \ell(t)$
        \vspace{.6em}
    \end{block}
    \vspace{1em}
    \begin{block}{Rule 2}
        \vspace{.6em}
        Freeze Kante wenn der Kostengewinn kleiner als $\epsilon OPT_t^{max}/ (t-\ell(t))$ ist.
        \vspace{.6em}
    \end{block}
\end{frame}

\begin{frame}
    \frametitle{\insertsubsection~In Worten}
    \underline{Finde Balance zwischen Anzahl Swaps $(\leq k)$ und Nutzen}\\
    \vspace{1em}
    \begin{block}{Rule 1}
        Verhindert das Swappen von Kanten, die zu einem Subgrafen gehören, der an den Gesamtkosten des MST keinen großen Einfluss hat.\\
        $\implies$Wurde vor / während der größten Iteration $\ell(t)$ geswapped. „Den besten bisherigen Subgrafen (und alles was schon davor gefroren war) fassen wir nicht mehr an.“
    \end{block}
    \vspace{1em}
    \begin{block}{Rule 2}
        Verhindert das Entfernen von Kanten deren Kosten kleiner als das Mittel seit der größten Iteration sind.\\
        $\implies$„Kein Swap, wenn direkter Nutzen zu gering”.
    \end{block}
\end{frame}


\subsection{Algorithmus Sequence Freeze}
\begin{frame}
    \frametitle{\insertsubsection}
    \underline{Wie simple Greedy Algorithmus mit Einschränkungen}\\
    \vspace{1em}
    Iteration $t$: $T_t = T_{t-1} \cup \{$„günstigste neue Kante”$\}$\\
    \vspace{1em}
    Betrachte Kanten $(f, h) \in (E_t \setminus T_t) \times T_t$\\
    Swap wenn nur, wenn..:
    \vspace{1em}
    \begin{itemize}
        \itemsep\setlength{1em}
        \item $(T_t \cup \{f\}) \setminus \{h\}$ bleibt ein Tree
    \end{itemize}
    \begin{enumerate}
        \itemsep\setlength{1em}
        \item $c(h) > (1+\epsilon) \cdot c(f)$
        \item $h = g_s^{i(s)}$ für ein $s \geq \ell(t) + 1$ (nicht gefroren durch \emph{Rule 1})
        \item $c(h) > \epsilon\frac{OPT_t^{max}}{t-\ell(t)}$ (nicht gefroren durch \emph{Rule 2})
    \end{enumerate}
\end{frame}

\subsection{Competitive Analysis (Skizze)}
\begin{frame}
    \frametitle{\insertsubsection}
    Competitiveness des Algoithmus im Vergleich zu OPT\\
    \vspace{.5em}
    \begin{block}{$(1+\epsilon)$-competitive}
        \vspace{.5em}
        \begin{itemize}
            \itemsep\setlength{.5em}
            \item \emph{Cond. 1} \& \emph{Cond. 3} -- Kosten steigen maximal um $(1+3\epsilon)$ pro Iteration $\rightarrow (1+O(\epsilon))$-competitive
            \item \emph{Cond. 2} -- Durch Weglassen „kostengünstiger” Swaps wird die Gesamtlösung im Vergleich zu $OPT_t$ maximal $O(\epsilon OPT_t)$ schlechter
        \end{itemize}
        \vspace{.1em}
    \end{block}
    \begin{itemize}
        \itemsep\setlength{.5em}
        \item Partitioniere Graf $T = T_{new} \cup T_{old}$ -- vor / nach längster Iteration
        \item $T_{old} := \{g_1^{i(1)}, ..., g_{\ell(t)}^{i(\ell(t))}\}$
        \item $T_{new} := \{g_{\ell(t)+1}^{i(\ell(t)+1)}, ..., g_t^{i(t)}\}$
    \end{itemize}
    Beweise Bounds für Partitionen
\end{frame}

\begin{frame}
    \frametitle{\insertsubsection}
    Amortisierter Upper Bound für Budget
    \vspace{.5em}
    \begin{itemize}
        \itemsep\setlength{.5em}
        \item $k_q := |T_q \setminus T_{q-1}|$: \underline{benutztes} Budget in Iteration $q$
        \item $D_\epsilon \in O(\frac{1}{\epsilon}\log\frac{1}{\epsilon})$
    \end{itemize}
    \vspace{.5em}
    \begin{block}{Zeige $\sum_{q=1}^t k_q \leq D_\epsilon \cdot t$}
        \vspace{.5em}
        \begin{enumerate}
            \itemsep\setlength{.8em}
            \item Upper Bound auf Swap Sequenz Länge\\
            \emph{Cond. 1} -- Nur Swap wenn Kostengewinn $> (1+\epsilon)$:\\Maximale Swapanzahl $\implies \log_{1+\epsilon}c(g_s^0) - \log_{1+\epsilon}c(g_s^{i(s)-1}) + 1$
            \item Lower Bound auf Swap Kosten\\
            \emph{Cond. 3} -- Nur Swap wenn Kosten d. zu entfernenden Kante Threshold übersteigt
        \end{enumerate}
        \vspace{.1em}
    \end{block}
\end{frame}

\subsection{Recap}
\begin{frame}
    \frametitle{\insertsection}
    \underline{Ergebnis}\\
    \vspace{1em}
    Online Algorithmus für MST mit folgenden bewiesenen Eigenschaften:\\
    \vspace{1em}
    \begin{itemize}
        \itemsep\setlength{.8em}
        \item $(1+\epsilon)$-competitive
        \item \underline{Amortisiertes} Budget $O(\frac{1}{\epsilon}\log\frac{1}{\epsilon})$
    \end{itemize}
    \vspace{1em}
    Da jeder online Algorithmus für MST mit obigen Eigenschaften mindestens ein Budget von $\Omega(\frac{1}{\epsilon})$ benötigt (vgl. Theorem~\ref{thm1}), liefert \emph{Sequence Freeze} das (logarithmisch) bestmögliche online Ergebnis.
\end{frame}
\section{Online MST mit striktem Budget}

\begin{frame}
    \frametitle{\insertsection}
    Kann ein striktes Budget besser sein? Insbesondere besser als $2$-competitive?
    \begin{block}{Gegenbeispiel}
        \vspace{.5em}
        \begin{itemize}
        \item Vollst. Graph, Knoten $v_0,...,v_n$
        \item $\forall t < n: c(v_t, v_n) = 1$
        \item $\forall s, t < n: c(v_s, v_t) = 2$
        \end{itemize}
        $\implies$ MST ist ein Stern um den letzten Knoten $n$\\
        \vspace{1em}
        Alle Iterationen $t < n$: $T_t$ enthält nur Kanten mit Kosten 2\\
        \vspace{1em}
        Bei fixem Budget $k$: $T_n$ enthält exakt $n-k$ Kanten der Kosten 2\\
        \vspace{1em}
        $\implies$ Competitive Ratio $2 - k/n \qquad (> 2 - \epsilon)$ bei großen $n$
    \end{block}
\end{frame}
\section{Anwendung auf TSP}

\subsection{Shortcutting (Beispiel)}

\begin{frame}
    \frametitle{\insertsubsection}
    \begin{block}{Eulerian Walk}
        \vspace{0.5em}
        \begin{itemize}
            \itemsep\setlength{0.7em}
            \item Besuche jede \emph{Kante} genau einmal
            \item Sei $2 \cdot T$ Multigraf zu T -- jede Kante verdoppelt\\$\implies$Jeder Knoten hat geraden Grad \\$\implies$Eulerian Walk möglich
        \end{itemize}
    \vspace{.8em}
    \end{block}
    \begin{itemize}
        \itemsep\setlength{0.7em}
        \item Bilde Eulerian Walk
        \item Notiere \emph{Knoten}reihenfolge
        \item Streiche alle mehrfach Vorkommnisse (\emph{Shortcut})\\$\implies$ Traveling Salesman Tour
    \end{itemize}
\end{frame}

\begin{frame}
    \frametitle{\insertsubsection}
    \begin{figure}
        \centering
        \includegraphics[width=1\linewidth]{./img/shortcutting.pdf}
        \caption{Tree (schwarz), Shortcut Tour (gestrichelt) \cite{recourse2016}}
    \end{figure}
    $W = u, v_1, w_1, \bar{v_1}, v_2, w_2, \bar{v_2}, v_3, w_3, \bar{v_3}, v_4, w_4, \bar{v_4}, \bar{\bar{v_3}}, \bar{\bar{v_2}}, \bar{\bar{v_1}}, \bar{u}$\\
    Tour $= u, v_1, w_1, v_2, w_2 ,v_3, w_3, v_4, w_4, (\bar{u})$
\end{frame}

\begin{frame}
    \frametitle{\insertsubsection}
    \begin{itemize}
        \itemsep\setlength{0.7em}
        \item Spanning Trees $R, R'$
        \item Unterschied nur 1 Kante: $R' = (R \cup \{f\}) \setminus \{g\}, f \notin R \wedge g \in R$\\$\implies$ „ein Swap Unterschied”
        \vspace{1em}
        \item Sehr ähnliche Bäume führen zu ggf. sehr unterschiedlichen TSP Touren
        \item Tour zu $R'$ (rechts): hat nie Knotenfolgen der Form $(w_i, v_{i+1})$\\$\implies$Bei steigender Knotenanzahl zunehmend großer Unterschied in Tour
    \end{itemize}
\end{frame}

\subsection{Robuste TSP Touren}

\begin{frame}
    \frametitle{\insertsubsection}
    \begin{theorem}[MST zu TSP]
        \vspace{.8em}
        Gegeben eine Reihe von metrischen Grafen $G_0, ..., G_t,..$ mit $G_t = (V_t, E_t) \wedge V_t = \{v_0, ...,v_t\}$. $T_t$ is Spanning Tree zu $G_t$ in Iteration $t$.\\
        \vspace{1em}
        Es existiert ein online Algorithmus, der zu jedem Teilgraf eine TSP Tour ausgibt ($Q_0, ..., Q_t, ...$), sodass
        \vspace{.7em}
        \begin{itemize}
            \itemsep\setlength{0.7em}
            \item $c(Q_t) \leq 2 \cdot c(T_t)$
            \item $|Q_t \setminus Q_{t-1}| \leq 4 \cdot |T_t \setminus T_{t-1}|$
        \end{itemize}
        \vspace{.8em}
    \end{theorem}
    \vspace{1em}
    \underline{Beweis benötigt verbessertes Shortcutting}
\end{frame}
\section{Zusammenfassung}

\begin{frame}
	\frametitle{\insertsection}
	\begin{itemize}
		\itemsep\setlength{1em}
		\item Standard online MST: Beste bekannte competitive Ratio $\Theta{\log n}$
		\item Neu Erkenntnis mit \cite{recourse2016}:\\Recourse verbessert die competitive Ratio zu \underline{$(1+\epsilon)$} (amortisiert)
		\item Verändertes Shortcutting Verfahren
		\item Online MST $\xrightarrow{\text{det. Verfahren}}$ online TSP\\
		Bewiesene Upper Bounds:
		\vspace{1em}
		\begin{itemize}
			\itemsep\setlength{.7em}
			\item Erhöht competitive Ratio um 2
			\item Erhöht Budget um 4
		\end{itemize}
		\item Online TSP $(2+\epsilon)$-competitive mit armortisiertem Budget $O(\frac{1}{\epsilon}\log\frac{1}{\epsilon})$
	\end{itemize}
\end{frame}

\section{Literatur}

\nocite{*}

\begin{frame}[allowframebreaks]{\insertsubsection}
    \begingroup
    \small
    \beamertemplatebookbibitems
    \bibliographystyle{apalike}
    \bibliography{bib}
    \endgroup
\end{frame}

\end{document}
