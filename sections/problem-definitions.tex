\section{Definitionen}
\subsection{MST und TSP}

\begin{frame}
    \frametitle{\insertsubsection}
    \begin{block}{Minimal Spanning Tree (MST)}
        \vspace{1em}
        \begin{itemize}
            \itemsep\setlength{.8em}
            \item Verbinde alle Knoten
            \item Azyklisch
        \end{itemize}
        Gesucht: kleinste (metrische) Verbindung aller Knoten
    \end{block}
    \vspace{1em}
    \begin{block}{Traveling Salesman Problem (TSP)}
        \vspace{1em}
        \begin{itemize}
            \itemsep\setlength{.8em}
            \item Besuche jeden Knoten einmal
            \item Zyklisch
        \end{itemize}
        Gesucht: kürzeste (metrische) Route
    \end{block}
\end{frame}

\begin{frame}
    \frametitle{\insertsubsection}
    Minimal Spanning Tree (MST)
    \begin{figure}
        \centering
        \includegraphics[width=.52\linewidth]{./img/mst.pdf}
        \caption{Minimal Spanning Tree\footnote{\url{https://commons.wikimedia.org/wiki/File:Minimum_spanning_tree.svg}}}
    \end{figure}
\end{frame}

\subsection{Online Problemstellung}

\begin{frame}
    \frametitle{\insertsubsection}
    „Nach und nach Knoten und zugehörige Kanten aufdecken. Problemstellung (MST bzw. TSP) in jedem Zug auf Teilgraf möglichst gut bearbeiten.“
    \vspace{1em}
    \begin{itemize}
        \itemsep\setlength{.8em}
        \item $\sigma = v_0, v_1, ...$ wobei $v_t$: \emph{Vertex}
        \item $\forall s \leq t-1: v_tv_s$ Kanten werden zusammen mit Knoten $t$ bekannt
        \item Kantengewichte erfüllen Dreiecksungleichung ($\rightarrow$ metrisch)
        \item Iteration $t$:
        \begin{itemize}
            \item Teilgraf $G_t = (V_t, E_t)$ 
            \item $V_t = \{v_0,...,v_t\}$
            \item $E_t = V_t \times V_t \rightarrow$ Graf ist „complete“
        \end{itemize}
    \end{itemize}
\end{frame}

\begin{frame}
    \frametitle{Online Ziele}
    \begin{block}{Online MST}
        \vspace{1em}
        \begin{itemize}
            \itemsep\setlength{.8em}
            \item Kanten Teilmenge $T_t$
            \item $T_t$ ist MST zu $G_t$
        \end{itemize}
    \end{block}
    \vspace{1em}
    \begin{block}{Online TSP}
        \vspace{1em}
        \begin{itemize}
            \itemsep\setlength{.8em}
            \item Tour durch $G_t$: $Q_t$
            \item $Q_t$ ist TSP Tour auf $G_t$ ($\approx$ minimiert traveling Kosten)
        \end{itemize}
    \end{block}
\end{frame}

%FIXME: definitionen von OPT und competitive ratio

\subsection{Einfacher Greedy Algorithmus}

\begin{frame}
    \frametitle{\insertsubsection}
    \begin{block}{Idee}
        „Bei neuem Knoten $v_t$ finde die Kante mit dem kleinsten Gewicht, die $v_t$ mit dem bisherigen Teilgrafen verbindet. Füge diese Kante zum MST hinzu.\\
        Nun iteriere, bis Stagnation: nimm eine kostengünstige Kante und tausche gegen eine teure Kante im MST.“
    \end{block}
\end{frame}

